%目的
本研究旨在探討建築營造行業如何藉由引入人機協作流程及數位雙生平台,作為發展全自動化機器人的過渡階段,協助蒐集人工智慧所需的訓練資料,解決該領域缺乏足夠現場數據之問題。

%方法
具體而言為確保預計開發之人機協作流程具自適應性和有效操控,本研究首先解析協作機器人在建築營造工地中所需的位移規劃資訊和操控需求,深入探討建築工地的特殊需求、機器人與人類工人的協作方式以及環境因素。通過分析資訊所獲得的結論,應用於後續開發自適應機器人系統,使其能夠在建築工地中進行協作操作。目前建築資訊模型主要用於工程專案管理,其模型的詳細程度有限,且檔案格式難以直接轉譯至機器人系統,因此限制了建築資訊模型在機器人控制領域的應用範圍和功能。為了克服這些限制,本研究將提出 4D BIM(包含時間資訊的建築資訊模型)至機器人系統的資訊轉換方法,並利用這些資訊來輔助人機協作系統的運作,藉此提升建築資訊模型在機器人領域的應用價值,使其能夠更好地支持機器人的控制、規劃和決策。透過建構一個數位雙生的人機協作平台,創建虛實整合環境,提供更準確和全面的資料,使人類工人和協作機器人能夠實時互動、合作和共享資訊。利用平台將整合實際感測器數據和虛擬分身,提供虛實整合的資訊以實現混合實境應用中的交互功能。通過收集現實建築工地和數位雙生環境中的相關資料,以供未來研究訓練智能模組,使機器人在現地能更好地理解和應對不同的工地場景和情境,逐步取代目前仍須人工介入的部分,增強機器人的自動化施工能力,促進營建工地發展全自動化機器人之應用。 

%結果
研究的主要貢獻針對移動機器人在油漆工程移動任務中進行了混合實境人機協作的測試。成功整合了 4D BIM 和輔助規劃資訊,將工項進度規劃納入協作式移動機器人系統中。在油漆工程的施作過程中,協作人員可以通過混合實境頭戴式裝置直觀地規劃任務、監督執行結果和指導協作設備。試驗結果顯示,這個流程能夠滿足工程現場移動施作油漆工程的需求,成功實現了協作式機器人的移動任務。此外還提供了一種保存智能行為數據的方式,以供未來研究用於訓練和優化人工智能模型,推動營建領域自動化和人工智慧應用的潛力。

\begin{flushleft}
\mbox{{\bf 關鍵字}: 人機協作、資料蒐集、建築資訊模型、自動化機器人、數位雙生}
\end{flushleft}
