\allowdisplaybreaks[4]

第五章內容

第一種方式是使用系統分配的序列埠名稱。當傳感器連接到 USB 接口,系統會自動分配一個名稱,例如 $/dev/ttyUSB0$、$/dev/ttyUSB1$、$/dev/ttyUSB2$ 等,以區分不同的串行設備。該方式簡單直觀,通過查看 $/dev$ 目錄下的檔案,即可確定硬體連接之序列埠名稱,然而當系統中存在多個串行設備時,需要進行名稱映射或手動配置,例如傳感器的連接順序發生變化或重新插拔,名稱可能發生變更,導致需要重新為設備手動調整配置。

第二種方式是使用 $/dev/serial/by-id$ 目錄下的序列埠名稱。每個串接設備都有一個獨特的唯一識別符號,系統根據此識別符號在 $/dev/serial/by-id$ 建立對應的連接埠名稱,確保了名稱的唯一性。這種方式不受設備連接順序的影響,即使重新插拔設備,名稱也不會改變。因此,本研究選擇使用基於唯一識別符號的方式來指定設備的序列埠,以維持系統的穩定性。 

\section{實驗場域}

為了驗證混合實境之人機協作資料蒐集流程的可行性,本研究利用成功大學某教學大樓之室內空間做為案例驗證場域,建築相關資訊如下:

\begin{itemize}
  \item 建物名稱:成功大學某教學大樓
  \item 建物規模:鋼筋混凝土建築結構,地下兩層、地上七層
  \item 案例驗證空間:2F 指定區域
\end{itemize}
